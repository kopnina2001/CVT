\documentclass[a4paper, 11pt]{article}

\usepackage{fullpage}

\usepackage[english,russian]{babel}
\usepackage[utf8]{inputenc}
\usepackage{amsmath, amsfonts, amssymb}
\usepackage{mathtools}
\usepackage{bm}

\usepackage{hyperref}
\usepackage{booktabs}
\usepackage{graphicx}
\usepackage{subcaption}

\usepackage[linesnumbered,ruled,vlined]{algorithm2e}
\SetKwInput{KwInput}{input}
\SetAlFnt{\small}

\usepackage{float}

\begin{document}

\begin{titlepage}
	\newpage
	
	\begin{center}
		Московский Государственный Университет им. М. В. Ломоносова \\
	\end{center}
	
	\vspace{8em}
	
	\begin{center}
		\Large Кафедра Вычислительных Технологий и Моделирования \\ 
	\end{center}
	
	\vspace{2em}
	
	\begin{center}
		\textsc{\textbf{Отчёт о практической работе: \linebreak решение стационарного уравнения диффузии \linebreak методом конечных объёмов \linebreak на равномерной сетке}}
	\end{center}
	
	\vspace{6em}
	
	
	
	\newbox{\lbox}
	\savebox{\lbox}{\hbox{}}
	\newlength{\maxl}
	\setlength{\maxl}{\wd\lbox}
	\hfill\parbox{11cm}{
		\hspace*{5cm}\hspace*{-5cm}Студент:\hfill\hbox {Иванов И. И.\hfill}\\
		\hspace*{5cm}\hspace*{-5cm}Преподаватель:\hfill\hbox {Ануприенко Д. В.\hfill}\\
		\\
		\hspace*{5cm}\hspace*{-5cm}Группа:\hfill\hbox {403}\\
	}
	
	
	\vspace{\fill}
	
	\begin{center}
		Москва \\ 2022
	\end{center}
	
\end{titlepage}

\setcounter{MaxMatrixCols}{20}


\section{Постановка задачи}

\subsection{Уравнение диффузии}

Рассмотрим стационарную, то есть, установившуюся со временем, диффузию некоторой примеси с концентрацией $C$, в двумерном случае.

Закон сохранения массы выражается уравнением 
\begin{equation}\label{eq:mass_balance}
\nabla \cdot \mathbf{q} = f,
\end{equation}
где $f$ -- функция источников/стоков, а поток $\mathbf{q} = [q_x~q_y]^T$ определяется законом Фика
\begin{equation}\label{eq:ficks_law}
\mathbf{q} = - \mathbb{D}\nabla C,
\end{equation}
где $\mathbb{D} = \mathbb{D}^T > 0$ -- тензор диффузии (матрица размера 2), определяющий диффузионные свойства среды в разных направлениях.

Совмещая закон сохранения массы \eqref{eq:mass_balance} и закон Фика \eqref{eq:ficks_law}, можно получить стационарное уравнение диффузии, в котором неизвестной является концентрация $C$:
\begin{equation}\label{eq:diffusion}
-\nabla\cdot \left(\mathbb{D}\nabla C\right) = f.
\end{equation}

Отметим, что уравнение \eqref{eq:diffusion} описывает не только диффузию, но и теплопроводность, течение подземных вод и другие процессы. В случае теплопроводности закон сохранения массы меняется на закон сохранения тепла, а закон Фика -- на закон Фурье, основной переменной становится температура. В случае течения подземных вод закон Фика меняется на закон Дарси, а переменной становится напор воды. 

\subsection{Краевая задача для уравнения диффузии}

Рассмотрим уравнение \eqref{eq:diffusion} в области $\Omega \subset \mathbb{R}^2$ с границей $\partial \Omega$, на которой зададим граничные условия Дирихле $C\big{|}_{\Gamma_D} = g_D$.
\subsection{Частный случай краевой задачи}

Будем решать задачу в квадратной области $\Omega = (0, 1) \times (0, 1)$.
Считаем, что тензор диффузии диагональный и постоянный во всей области:
\begin{equation}
\mathbb{D} = \begin{bmatrix} d_x & 0 \\ 0 & d_y \end{bmatrix}.
\end{equation}
Такая форма тензора говорит о том, что диффузия в горизонтальном и вертикальном направлениях протекает со своими коэффициентами. В этом случае уравнение \eqref{eq:diffusion} принимает следующий вид
\begin{equation}
- d_x \dfrac{\partial^2 C}{\partial^2 x} - d_y \dfrac{\partial^2 C}{\partial^2 y} = f.
\end{equation}


Граничные условия и правая часть задаются таким образом, чтобы решением являлась функция 
$C(x, y) = \sin(\pi x) \sin(\pi y)$. Отсюда
$$f(x, y) = (d_x + d_y) \sin(\pi x) \sin(\pi y)$$, а
$$g_D(x, y) = \sin(\pi x) \sin(\pi y).
$$


\section{Численное решение}

\subsection{Сетка}
Введём в области $\Omega$ равномерную по обеим координатам сетку: 
$x_i = i \cdot h$, $y_j = j \cdot h$, $i, j \in \overline{0, N - 1}$, $h = \dfrac{1}{N}$, $N$ 
- количество отрезков на каждой стороне квадрата.

\subsection{Метод конечных разностей}

Описание метода, принципов, неизвестных, общей системы уравнений

\subsection{Решение линейной системы}
Для решения линейной системы использовался прямой метод (тип метода) / итерационный метод (тип метода, тип и параметры переобуславливателя)

\section{Численные результаты}

Реализация алгоритмов была проведена на языке Я с помощью библиотек Б1, Б2. Для исследования сходимости метода искались $C$- и $L_2$-нормы ошибки, вычисляемые по формулам
$$\|err\|_C = \max_{i, j \in \overline{0, N - 1}} \left| C_{ij} - C(x_i, y_j)\right|,$$
$$\|err\|_{L_2} = \left( \sum_{i = 0}^{N - 1}\sum_{j = 0}^{N - 1} \int_{e_{ij}} (C_h(x,y) - C(x, y))^2 dxdy \right)^{\frac{1}{2}}.$$

Отметим, что в случае $L_2$-нормы необходимо считать интегралы по ячейкам $e_{ij}$ от разности точного решения $C$ и приближенного решения $C_h$. При этом МКР не предполагает какого-то конкретного вида функции $C_h$, давая лишь значения этой функции в узлах. Поэтому предполагаем $C_h$ кусочно-билинейной функцией.

Результаты расчётов приведены в таблице ниже.
\begin{center}
	\begin{tabular}{ | c | l | l | c | }
		\hline
		N & $\|err\|_C$ & $\|err\|_{L_2}$ & Число итераций (для итер. метода) \\ \hline
		%4 & $1.28 \cdot 10^{-1}$ & $1.57 \cdot 10^{-1}$ & 4 \\ \hline
		%8 & $3.34 \cdot 10^{-2}$ & $7.97 \cdot 10^{-2}$ & 20 \\ \hline
		%16 & $8.40 \cdot 10^{-3}$ & $4.00 \cdot 10^{-2}$ & 47 \\ \hline
		%32 & $2.10 \cdot 10^{-3}$ & $2.00 \cdot 10^{-2}$ & 150 \\ \hline
		%64 & $5.22 \cdot 10^{-4}$ & $1.00 \cdot 10^{-2}$ & 358 \\ \hline
		%128 & $1.27 \cdot 10^{-4}$ & $5.01 \cdot 10^{-3}$ & 1232 \\ \hline
	\end{tabular}
\end{center}

Из графика (рисунок \ref{fig:order}) логарифмической зависимости $N$, $\|err\|_C$ и $\|err\|_{L_2}$ можем сделать вывод, что метод сходится со вторым порядком точности по $h$ в норме $C$ и с первым порядком по $h$ в норме $L_2$.

\begin{figure}[h] \centering
	\includegraphics[scale=1]{order.png}
	\caption{Пример рисунка, на котором представлены графики ошибки, логарифмическая шкала используется для обеих осей. Наглядно показано, что присутствует второй порядок ошибки. \textbf{ВАЖНО!} Данный рисунок взят из другой задачи, здесь должны быть несколько другие порядки\label{fig:order}}
\end{figure}

Ниже на рисунке ... приведены графики точного решения, приближенного решения, модуля ошибки для двух-трех сеток.


\section{Выводы}

В рамках практической работы изучен метод конечных разностей на равномерной сетке для решения краевой задачи Дирихле для стационарного уравнения диффузии в квадратной области из $\mathbb{R}^2$. \dots Метод сходится по шагу сетки с ... порядком точности в норме $C$ и с ... порядком в норме $L_2$, что соответствует теоретическим оценкам.


\end{document}